% TODO
% mention lossless <- simplifies...
% example of need of virt channels
% more topologies





%% bare_conf.tex
%% V1.3
%% 2007/01/11
%% by Michael Shell
%% See:
%% http://www.michaelshell.org/
%% for current contact information.
%%
%% This is a skeleton file demonstrating the use of IEEEtran.cls
%% (requires IEEEtran.cls version 1.7 or later) with an IEEE conference paper.
%%
%% Support sites:
%% http://www.michaelshell.org/tex/ieeetran/
%% http://www.ctan.org/tex-archive/macros/latex/contrib/IEEEtran/
%% and
%% http://www.ieee.org/

%%*************************************************************************
%% Legal Notice:
%% This code is offered as-is without any warranty either expressed or
%% implied; without even the implied warranty of MERCHANTABILITY or
%% FITNESS FOR A PARTICULAR PURPOSE! 
%% User assumes all risk.
%% In no event shall IEEE or any contributor to this code be liable for
%% any damages or losses, including, but not limited to, incidental,
%% consequential, or any other damages, resulting from the use or misuse
%% of any information contained here.
%%
%% All comments are the opinions of their respective authors and are not
%% necessarily endorsed by the IEEE.
%%
%% This work is distributed under the LaTeX Project Public License (LPPL)
%% ( http://www.latex-project.org/ ) version 1.3, and may be freely used,
%% distributed and modified. A copy of the LPPL, version 1.3, is included
%% in the base LaTeX documentation of all distributions of LaTeX released
%% 2003/12/01 or later.
%% Retain all contribution notices and credits.
%% ** Modified files should be clearly indicated as such, including  **
%% ** renaming them and changing author support contact information. **
%%
%% File list of work: IEEEtran.cls, IEEEtran_HOWTO.pdf, bare_adv.tex,
%%                    bare_conf.tex, bare_jrnl.tex, bare_jrnl_compsoc.tex
%%*************************************************************************

% *** Authors should verify (and, if needed, correct) their LaTeX system  ***
% *** with the testflow diagnostic prior to trusting their LaTeX platform ***
% *** with production work. IEEE's font choices can trigger bugs that do  ***
% *** not appear when using other class files.                            ***
% The testflow support page is at:
% http://www.michaelshell.org/tex/testflow/



% Note that the a4paper option is mainly intended so that authors in
% countries using A4 can easily print to A4 and see how their papers will
% look in print - the typesetting of the document will not typically be
% affected with changes in paper size (but the bottom and side margins will).
% Use the testflow package mentioned above to verify correct handling of
% both paper sizes by the user's LaTeX system.
%
% Also note that the "draftcls" or "draftclsnofoot", not "draft", option
% should be used if it is desired that the figures are to be displayed in
% draft mode.
%
\documentclass[conference]{IEEEtran}
% Add the compsoc option for Computer Society conferences.
%
% If IEEEtran.cls has not been installed into the LaTeX system files,
% manually specify the path to it like:
% \documentclass[conference]{../sty/IEEEtran}





% Some very useful LaTeX packages include:
% (uncomment the ones you want to load)


% *** MISC UTILITY PACKAGES ***
%
%\usepackage{ifpdf}
% Heiko Oberdiek's ifpdf.sty is very useful if you need conditional
% compilation based on whether the output is pdf or dvi.
% usage:
% \ifpdf
%   % pdf code
% \else
%   % dvi code
% \fi
% The latest version of ifpdf.sty can be obtained from:
% http://www.ctan.org/tex-archive/macros/latex/contrib/oberdiek/
% Also, note that IEEEtran.cls V1.7 and later provides a builtin
% \ifCLASSINFOpdf conditional that works the same way.
% When switching from latex to pdflatex and vice-versa, the compiler may
% have to be run twice to clear warning/error messages.






% *** CITATION PACKAGES ***
%
%\usepackage{cite}
% cite.sty was written by Donald Arseneau
% V1.6 and later of IEEEtran pre-defines the format of the cite.sty package
% \cite{} output to follow that of IEEE. Loading the cite package will
% result in citation numbers being automatically sorted and properly
% "compressed/ranged". e.g., [1], [9], [2], [7], [5], [6] without using
% cite.sty will become [1], [2], [5]--[7], [9] using cite.sty. cite.sty's
% \cite will automatically add leading space, if needed. Use cite.sty's
% noadjust option (cite.sty V3.8 and later) if you want to turn this off.
% cite.sty is already installed on most LaTeX systems. Be sure and use
% version 4.0 (2003-05-27) and later if using hyperref.sty. cite.sty does
% not currently provide for hyperlinked citations.
% The latest version can be obtained at:
% http://www.ctan.org/tex-archive/macros/latex/contrib/cite/
% The documentation is contained in the cite.sty file itself.






% *** GRAPHICS RELATED PACKAGES ***
%
\ifCLASSINFOpdf
  \usepackage[pdftex]{graphicx}
  % declare the path(s) where your graphic files are
  % \graphicspath{{../pdf/}{../jpeg/}}
  % and their extensions so you won't have to specify these with
  % every instance of \includegraphics
  % \DeclareGraphicsExtensions{.pdf,.jpeg,.png}
\else
  % or other class option (dvipsone, dvipdf, if not using dvips). graphicx
  % will default to the driver specified in the system graphics.cfg if no
  % driver is specified.
  % \usepackage[dvips]{graphicx}
  % declare the path(s) where your graphic files are
  % \graphicspath{{../eps/}}
  % and their extensions so you won't have to specify these with
  % every instance of \includegraphics
  % \DeclareGraphicsExtensions{.eps}
\fi
% graphicx was written by David Carlisle and Sebastian Rahtz. It is
% required if you want graphics, photos, etc. graphicx.sty is already
% installed on most LaTeX systems. The latest version and documentation can
% be obtained at: 
% http://www.ctan.org/tex-archive/macros/latex/required/graphics/
% Another good source of documentation is "Using Imported Graphics in
% LaTeX2e" by Keith Reckdahl which can be found as epslatex.ps or
% epslatex.pdf at: http://www.ctan.org/tex-archive/info/
%
% latex, and pdflatex in dvi mode, support graphics in encapsulated
% postscript (.eps) format. pdflatex in pdf mode supports graphics
% in .pdf, .jpeg, .png and .mps (metapost) formats. Users should ensure
% that all non-photo figures use a vector format (.eps, .pdf, .mps) and
% not a bitmapped formats (.jpeg, .png). IEEE frowns on bitmapped formats
% which can result in "jaggedy"/blurry rendering of lines and letters as
% well as large increases in file sizes.
%
% You can find documentation about the pdfTeX application at:
% http://www.tug.org/applications/pdftex





% *** MATH PACKAGES ***
%
%\usepackage[cmex10]{amsmath}
% A popular package from the American Mathematical Society that provides
% many useful and powerful commands for dealing with mathematics. If using
% it, be sure to load this package with the cmex10 option to ensure that
% only type 1 fonts will utilized at all point sizes. Without this option,
% it is possible that some math symbols, particularly those within
% footnotes, will be rendered in bitmap form which will result in a
% document that can not be IEEE Xplore compliant!
%
% Also, note that the amsmath package sets \interdisplaylinepenalty to 10000
% thus preventing page breaks from occurring within multiline equations. Use:
%\interdisplaylinepenalty=2500
% after loading amsmath to restore such page breaks as IEEEtran.cls normally
% does. amsmath.sty is already installed on most LaTeX systems. The latest
% version and documentation can be obtained at:
% http://www.ctan.org/tex-archive/macros/latex/required/amslatex/math/





% *** SPECIALIZED LIST PACKAGES ***
%
%\usepackage{algorithmic}
% algorithmic.sty was written by Peter Williams and Rogerio Brito.
% This package provides an algorithmic environment fo describing algorithms.
% You can use the algorithmic environment in-text or within a figure
% environment to provide for a floating algorithm. Do NOT use the algorithm
% floating environment provided by algorithm.sty (by the same authors) or
% algorithm2e.sty (by Christophe Fiorio) as IEEE does not use dedicated
% algorithm float types and packages that provide these will not provide
% correct IEEE style captions. The latest version and documentation of
% algorithmic.sty can be obtained at:
% http://www.ctan.org/tex-archive/macros/latex/contrib/algorithms/
% There is also a support site at:
% http://algorithms.berlios.de/index.html
% Also of interest may be the (relatively newer and more customizable)
% algorithmicx.sty package by Szasz Janos:
% http://www.ctan.org/tex-archive/macros/latex/contrib/algorithmicx/




% *** ALIGNMENT PACKAGES ***
%
%\usepackage{array}
% Frank Mittelbach's and David Carlisle's array.sty patches and improves
% the standard LaTeX2e array and tabular environments to provide better
% appearance and additional user controls. As the default LaTeX2e table
% generation code is lacking to the point of almost being broken with
% respect to the quality of the end results, all users are strongly
% advised to use an enhanced (at the very least that provided by array.sty)
% set of table tools. array.sty is already installed on most systems. The
% latest version and documentation can be obtained at:
% http://www.ctan.org/tex-archive/macros/latex/required/tools/


%\usepackage{mdwmath}
%\usepackage{mdwtab}
% Also highly recommended is Mark Wooding's extremely powerful MDW tools,
% especially mdwmath.sty and mdwtab.sty which are used to format equations
% and tables, respectively. The MDWtools set is already installed on most
% LaTeX systems. The lastest version and documentation is available at:
% http://www.ctan.org/tex-archive/macros/latex/contrib/mdwtools/


% IEEEtran contains the IEEEeqnarray family of commands that can be used to
% generate multiline equations as well as matrices, tables, etc., of high
% quality.


%\usepackage{eqparbox}
% Also of notable interest is Scott Pakin's eqparbox package for creating
% (automatically sized) equal width boxes - aka "natural width parboxes".
% Available at:
% http://www.ctan.org/tex-archive/macros/latex/contrib/eqparbox/





% *** SUBFIGURE PACKAGES ***
%\usepackage[tight,footnotesize]{subfigure}
% subfigure.sty was written by Steven Douglas Cochran. This package makes it
% easy to put subfigures in your figures. e.g., "Figure 1a and 1b". For IEEE
% work, it is a good idea to load it with the tight package option to reduce
% the amount of white space around the subfigures. subfigure.sty is already
% installed on most LaTeX systems. The latest version and documentation can
% be obtained at:
% http://www.ctan.org/tex-archive/obsolete/macros/latex/contrib/subfigure/
% subfigure.sty has been superceeded by subfig.sty.



%\usepackage[caption=false]{caption}
\usepackage[font=footnotesize]{subfig}
% subfig.sty, also written by Steven Douglas Cochran, is the modern
% replacement for subfigure.sty. However, subfig.sty requires and
% automatically loads Axel Sommerfeldt's caption.sty which will override
% IEEEtran.cls handling of captions and this will result in nonIEEE style
% figure/table captions. To prevent this problem, be sure and preload
% caption.sty with its "caption=false" package option. This is will preserve
% IEEEtran.cls handing of captions. Version 1.3 (2005/06/28) and later 
% (recommended due to many improvements over 1.2) of subfig.sty supports
% the caption=false option directly:
%\usepackage[caption=false,font=footnotesize]{subfig}
%
% The latest version and documentation can be obtained at:
% http://www.ctan.org/tex-archive/macros/latex/contrib/subfig/
% The latest version and documentation of caption.sty can be obtained at:
% http://www.ctan.org/tex-archive/macros/latex/contrib/caption/




% *** FLOAT PACKAGES ***
%
%\usepackage{fixltx2e}
% fixltx2e, the successor to the earlier fix2col.sty, was written by
% Frank Mittelbach and David Carlisle. This package corrects a few problems
% in the LaTeX2e kernel, the most notable of which is that in current
% LaTeX2e releases, the ordering of single and double column floats is not
% guaranteed to be preserved. Thus, an unpatched LaTeX2e can allow a
% single column figure to be placed prior to an earlier double column
% figure. The latest version and documentation can be found at:
% http://www.ctan.org/tex-archive/macros/latex/base/



%\usepackage{stfloats}
% stfloats.sty was written by Sigitas Tolusis. This package gives LaTeX2e
% the ability to do double column floats at the bottom of the page as well
% as the top. (e.g., "\begin{figure*}[!b]" is not normally possible in
% LaTeX2e). It also provides a command:
%\fnbelowfloat
% to enable the placement of footnotes below bottom floats (the standard
% LaTeX2e kernel puts them above bottom floats). This is an invasive package
% which rewrites many portions of the LaTeX2e float routines. It may not work
% with other packages that modify the LaTeX2e float routines. The latest
% version and documentation can be obtained at:
% http://www.ctan.org/tex-archive/macros/latex/contrib/sttools/
% Documentation is contained in the stfloats.sty comments as well as in the
% presfull.pdf file. Do not use the stfloats baselinefloat ability as IEEE
% does not allow \baselineskip to stretch. Authors submitting work to the
% IEEE should note that IEEE rarely uses double column equations and
% that authors should try to avoid such use. Do not be tempted to use the
% cuted.sty or midfloat.sty packages (also by Sigitas Tolusis) as IEEE does
% not format its papers in such ways.





% *** PDF, URL AND HYPERLINK PACKAGES ***
%
%\usepackage{url}
% url.sty was written by Donald Arseneau. It provides better support for
% handling and breaking URLs. url.sty is already installed on most LaTeX
% systems. The latest version can be obtained at:
% http://www.ctan.org/tex-archive/macros/latex/contrib/misc/
% Read the url.sty source comments for usage information. Basically,
% \url{my_url_here}.





% *** Do not adjust lengths that control margins, column widths, etc. ***
% *** Do not use packages that alter fonts (such as pslatex).         ***
% There should be no need to do such things with IEEEtran.cls V1.6 and later.
% (Unless specifically asked to do so by the journal or conference you plan
% to submit to, of course. )


% correct bad hyphenation here
\hyphenation{op-tical net-works semi-conduc-tor}


\begin{document}
%
% paper title
% can use linebreaks \\ within to get better formatting as desired
\title{A Transport-Layer Network for Distributed FPGA Platforms}


% author names and affiliations
% use a multiple column layout for up to three different
% affiliations
%\author{\IEEEauthorblockN{Michael Shell}
%\IEEEauthorblockA{School of Electrical and\\Computer Engineering\\
%Georgia Institute of Technology\\
%Atlanta, Georgia 30332--0250\\
%Email: http://www.michaelshell.org/contact.html}
%\and
%\IEEEauthorblockN{Homer Simpson}
%\IEEEauthorblockA{Twentieth Century Fox\\
%Springfield, USA\\
%Email: homer@thesimpsons.com}
%\and
%\IEEEauthorblockN{James Kirk\\ and Montgomery Scott}
%\IEEEauthorblockA{Starfleet Academy\\
%San Francisco, California 96678-2391\\
%Telephone: (800) 555--1212\\
%Fax: (888) 555--1212}}

% conference papers do not typically use \thanks and this command
% is locked out in conference mode. If really needed, such as for
% the acknowledgment of grants, issue a \IEEEoverridecommandlockouts
% after \documentclass

% for over three affiliations, or if they all won't fit within the width
% of the page, use this alternative format:
% 
%\author{\IEEEauthorblockN{Michael Shell\IEEEauthorrefmark{1},
%Homer Simpson\IEEEauthorrefmark{2},
%James Kirk\IEEEauthorrefmark{3}, 
%Montgomery Scott\IEEEauthorrefmark{3} and
%Eldon Tyrell\IEEEauthorrefmark{4}}
%\IEEEauthorblockA{\IEEEauthorrefmark{1}School of Electrical and Computer Engineering\\
%Georgia Institute of Technology,
%Atlanta, Georgia 30332--0250\\ Email: see http://www.michaelshell.org/contact.html}
%\IEEEauthorblockA{\IEEEauthorrefmark{2}Twentieth Century Fox, Springfield, USA\\
%Email: homer@thesimpsons.com}
%\IEEEauthorblockA{\IEEEauthorrefmark{3}Starfleet Academy, San Francisco, California 96678-2391\\
%Telephone: (800) 555--1212, Fax: (888) 555--1212}
%\IEEEauthorblockA{\IEEEauthorrefmark{4}Tyrell Inc., 123 Replicant Street, Los Angeles, California 90210--4321}}




% use for special paper notices
%\IEEEspecialpapernotice{(Invited Paper)}




% make the title area
\maketitle

\begin{abstract}
%\boldmath
We present a transport-layer network that aids developers in building safe,
high-performance distributed FPGA applications. Two essential features of such a
network are virtual channels and end-to-end flow control. Since different
virtual channels can have vastly different traffic patterns, a proper network
design requires flexibility in setting buffer sizes and flow control credits. In
addition, the protocol must have very low latency and low memory resource
requirements, because the communication links between FPGAs have very low
latency, and FPGAs have limited on-chip memory. These resource requirements make
protocols such as TCP/IP unsuitable in this environment. Our network implements
these features, and takes advantage of the low error characteristic of a rack
level network deployment to implement a low overhead credit based end-to-end
flow control. Our design has many parameters in the source code which can be set
at the time of FPGA synthesis. 

Our prototype cluster, which is composed of 20 Xilinx VC707 boards, each with 4
20Gb/s serial links, achieves effective bandwidth of 85\% of the maximum
physical bandwidth, and a latency of 0.5us per hop.  Our network exposes a
variable width FIFO channel abstraction, with the ability to adjust buffer size
and flow control credits per channel. Several applications have already been
developed using this network. The user feedback suggest that these features make
application development significantly easier.

%Many modern large-scale data-intensive applications can benefit from the use of
%distributed FPGAs, which offer high performance and low power consumption.
%Networking a cluster of FPGAs using generic interconnect technologies such as Ethernet and
%TCP/IP is usually difficult because of the high resource and management overhead
%of such technologies. Instead, they are often networked using a low overhead link
%level protocol using the on-chip multi-gigabit serial transceivers. Due to the
%engineering cost and on-chip resource limitations, most
%existing network implementations using these links provide a very low level
%interface, providing packet routing but rarely higher level features such as
%virtual channels or end-to-end flow control. Developing a correct distributed
%application using such a network interface is not easy for most developers, and
%this is one of the major hurdles of large scale distributed FPGA application
%development.
%
%This paper presents a transport-layer network infrastructure that aids
%application developers to build safe, high-performance distributed FPGA
%applications. The network implements convenient features such as virtual
%channels, and takes advantage of the low error characteristic of a rack level
%network deployment to implement a low overhead credit based end-to-end flow
%control. The network is parameterized so that each virtual channel can have
%different, optimized flow control characteristics. 
%Our prototype implementation achieves an effective
%bandwidth of 17Gb/s per link, which is 85\% of the maximum physical bandwidth, and a
%latency of 0.5us.  Out network exposes a variable width FIFO abstraction, which
%is convenient for application developers.

\end{abstract}



% IEEEtran.cls defaults to using nonbold math in the Abstract.
% This preserves the distinction between vectors and scalars. However,
% if the conference you are submitting to favors bold math in the abstract,
% then you can use LaTeX's standard command \boldmath at the very start
% of the abstract to achieve this. Many IEEE journals/conferences frown on
% math in the abstract anyway.

% no keywords




% For peer review papers, you can put extra information on the cover
% page as needed:
% \ifCLASSOPTIONpeerreview
% \begin{center} \bfseries EDICS Category: 3-BBND \end{center}
% \fi
%
% For peerreview papers, this IEEEtran command inserts a page break and
% creates the second title. It will be ignored for other modes.
\IEEEpeerreviewmaketitle


\section{Introduction}
\label{sec:intro}

Google has predicted flu outbreaks by analyzing social network information a
week faster than CDC~\cite{googleflu}. Analysis of twitter data can reveal social
upheavals faster than journalists. Amazon is planning to use customer data for
preemptive shipping of products. Real-time analysis of personal genome may
significantly aid in diagnostics. Big Data analytics are potentially going to
have revolutionary impact on the way scientific discoveries are made. By many
accounts, complex analysis of Big Data is going to be the biggest economic
driver for the IT industry.

Big Data by definition doesn’t fit in personal computers or DRAM of even
moderate size clusters. Since the data may be stored on hard disks, latency and
throughput of storage access is of primary concern. Historically, this has been
mitigated by organizing the processing of data in a highly sequential manner.
However, complex queries cannot always be organized for sequential data
accesses, and thus high performance implementations of such queries pose a
great challenge. One approach to solving this problem is \emph{ram
cloud}~\cite{ramcloud}, where the cluster has enough collective DRAM to accommodate the
entire dataset in DRAM. In this paper, we explore much cheaper alternatives
where Big Data analytics can be done with reasonable efficiency in a single
rack.

Another alternative to speed up complex data analysis is to use SSDs instead of
disks because of superior performance of flash devices in terms of latency of
access and throughput, especially for random accesses. SSDs have been developed
with the goal to be a drop-in replacement of hard disks. This has led to their
widespread use especially in embedded space, but the use of this legacy
interface has resulted in suboptimal use of flash capabilities. The
sub-optimality arises both from the use of a \emph{Flash Translation Layer}
(FTL), and the use of old hardware interfaces like SATA. The latter is somewhat
mitigated by the use of newer interfaces like PCIe.  High-performance enterprise
SSDs such as FusionIO~\cite{fusionio}, Violinmemory~\cite{violinmemory} and Intel NVMe
devices~\cite{intelnvme} solve many of these issues by implementing an improved interface
such as NVMe over PCIe. Attempts to remove the FTL and let the database make
high level decisions~\cite{noftl} have shown to be beneficial, but such
solutions are not yet widely adopted.


The latency to access storage over Ethernet in a hard-disk-based cluster is
dominated by the latency of the hard disk itself. However, this is not the case
in an SSD-based cluster, where network latency may even be larger than the SSD
latency. At this low-level of latency, even software stack overhead becomes a
significant concern. These concerns have been addressed by faster network
fabrics such as 10Gb Ethernet and Infiniband~\cite{infiniband}, and by low-overhead
software protocols such as RDMA~\cite{rdmampi} or user-level TCP stacks that bypass the
operating system~\cite{usertcp}. QuickSAN integrates a network interface into the
storage device, thus removing a layer of software overhead.

Another solution to the network performance problem is to reduce the network
traffic by in-store computing. For example, if one can perform some filtering
operation in the disk controller without bringing it to the host, it may
dramatically reduce the amount of data that needs to be transferred between the
disk and the host~\cite{idisk,netezza,smartssdquery}. Even though this idea has been
around for a long time, it has not found much traction, perhaps because the
characteristics of disks masked the benefits of this approach.  

In this paper, we present FlashBoost, a system designed to address all of the
aforementioned problems in the context of complex analysis of Big Data. Our goal
is to provide a rack-level system which can address many Big Data problems whose
datasets are 10\textasciitilde20 TB. Specifically, we have implemented a system which has 20
nodes where each node consists of a server with 12 Xeon cores, 48 GB of DRAM and
2 TB of disk. We have augmented each server with a Xilinx VC707 FPGA board and
1TB of flash on a custom design board. The FPGA board is connected to the server
via PCIe on one side, and to the flash card by 8 lanes of 6.6Gbps serial
communication links. All FPGAs can be connected to each other in various
topologies using 8 10Gbps serial links provided by the FPGA. The overall
architecture can be seen in Figure~\ref{fig:architecture}.

The system hardware and software provides the following capabilities:
\begin{enumerate}
\item Large enough storage to host Big Data workloads in the 10\textasciitilde20 TB range
\item Near-uniform latency access into a network of storage devices that form a
global address space
\item Capacity to implement user-defined in-storage processing engines in the
embedded FPGA
\item Flash card design which exposes a low-latency interface with details to
exploit parallelism in flash chip accesses by in-storage processing engines
\end{enumerate}

Our preliminary experimental results show that FlashBoost performance is
significantly better than (1) a disk-based system and (2) a flash-based system
in which flash is used as a disk replacement. We also show that accelerators can
provide a factor of XXX performance benefits over a non-accelerated system.
FlashBoost unambiguously establishes an architecture whose
price-performance-power characteristics provide an attractive alternative for
doing similar scale applications in a ram cloud.

The main contributions of this work are: (1) Design and implementation of a
scalable flash-based system with a global address space and in-store computing
capability. (2) A hardware-software codesign environment for incorporating
user-defined in-store processing engines. (3) Performance measurements that show
the advantage of such an architecture over using flash as a drop-in replacement
for disks. (4) Demonstration of a complex data analytics appliance which is much
cheaper and consumes an order of magnitude less power than the cloud-based
alternative.

The rest of the paper is organized as follows: In section~\ref{sec:related} we
explore some existing research related to our system. In
section~\ref{sec:architecture} we describe the architecture of our rack-level
system, and in section~\ref{sec:software} we describe the software interface
that can be used to access flash and the accelerators. In
section~\ref{sec:acceleration} we describe some example accelerators we have
built for the FlashBoost system. In section~\ref{sec:implementation} we describe
our hardware implementation of FlashBoost, and show our results from the
implementation in section~\ref{sec:results} and \ref{sec:results_acceleration}.




\section{Related Work}

%\item Computation/Storage coupling (computation near data), for MapReduce, etc
%\item Flash-based databases (Zetascale)
Flash-based databased have been under active investigation as an way to expand
the working set beyond what is feasible using relatively expensive DRAM.
Flash-based solutions such as SanDisk ZetaScale~\ref{zetascale} have
demonstrated that by optimizing databases for flash access, a
performance-per-dollar much higher that with DRAM-based machines can be
achieved.

As flash storage became faster, and the bottleneck of system performance became
other components such as the storage area network, attempts to reduce the
network software overhead or even providing a separate sideband network is being
explored. One such solution is QuickSAN~\ref{quicksan}, which implements much of
the network protocol in a reconfigurable hardware fabric.

FPGA-based application-specific hardware accelerators is becoming more popular
for HPC workloads due to its high performance and low power consumption.
Applications such as linear algebra operations~\ref{}, genome sequencing~\ref{}
and high-speed data filtering~\ref{} has been successful, and even general
purpose database accelerators have been built and commercialized~\ref{netezza}.
Microsoft Research recently has demonstrated the power/performance
benefits of using FPGAs in a cluster, in the context of accelerating Bing
search~\ref{msr}. In their work, each server in a cluster were augmented with an FPGA, and
the FPGAs were networked to each other using high-speed serial links.
However, such accelerator appliances depend on the host streaming data to the
accelerator, and therefore are only effective for datasets that can
still fit in the combined DRAM capacity of host machines. Once the data doesn't
fit, data needs to be streamed from secondary storage, which ends up being the
bottleneck.

Another way hardware accelerators can be deployed is an embedded processor in
the datapath of normal data access, such as in network and storage interfaces. 
An accelerator with an implementation of the
network protocol can plug directly into a network
port, and process data as it is being passed across the network link. This
removes much of the data transport and protocol overhead and allows very low
latency operations. This idea has often been demonstrated in financial
applications that require extremely low latency, and also in latency-sensitive
applications such as memcached~\ref{memcached}. Accelerators such as compression
or filtering accelerators in the network or storage can help reduce the amount
of data that need to be sent over a link, effectively amplifying its bandwidth.

\section{System Architecture}

The FlashBoost architecture is composed of multiple identical nodes. Each node
consists of a host server and a FlashBoost storage device. The host servers are
networked together using Ethernet or other general-purpose networking fabric.
The FlashBoost storage device consists of a in-store processing engine, flash
storage and flash controller, host interface, and a high-speed network controller.
The in-store processing engine can implement a raw
access to flash and network resources, an application-specific hardware
accelerator, or both. The host software can send high-level requests over the
PCIe link to the in-storage processing engine to access flash network and computation resources.
In our implementation of FlashBoost, we have used a Field Programmable Gate
Array (FPGA) to implement The in-store processor and also the flash, host and
network controllers. However, the FlashBoost Architecture should not be limited
to an FPGA-based implementation.


\begin{figure*}[ht]
	\begin{center}
	\includegraphics[width=0.8\paperwidth]{figures/architecture.pdf}
	\caption{FlashBoost Architecture}
	\label{fig:architecture}
	\end{center}
\end{figure*}


The in-store processing engine has access to four major services, The flash
controller, network controller, host interface and the DRAM Buffer.
Figure~\ref{fig:ispservice} shows the four services. Development of FlashBoost was
done in the high-level hardware description language Bluespec. As a result, all
services expose a high-level language interface using latency-insensitive FIFOs
for communication. This makes the services intuitive to use, and flexible to be
used easily with many in-storage processing engines.

\begin{figure}[h]
	\begin{center}
	\includegraphics[width=0.3\paperwidth]{figures/isp-service-crop.pdf}
	\caption{Services Provided to In-Store Processor}
	\label{fig:ispcore}
	\end{center}
\end{figure}

\subsection{Flash Interface}

The near-data processing core implements a error-free low-level interface into
the flash storage. (Ming)

\begin{figure}[h]
	\begin{center}
	\includegraphics[scale=0.4]{figures/top-arch-crop.pdf}
	\caption{Flash Interface}
	\label{fig:flashinterface}
	\end{center}
\end{figure}

\subsection{Inter-Storage Network}

Conventional general-purpose networking infrastructures such as TCP are designed to serve
all kinds of network traffic over any distance, ranging from intra-rack to over
continents. This makes the protocol incredibly complex and results in a large
protocol overhead.

The near-data processors in FlashBoost are linked together using a separate
high-performance network among themselves in which all nodes have multiple
network ports and act both as a network switch as well as an endpoint. This network provides low-level
routing, some flow control and use of virtual channels, but omits various high level
features that modern network protocols provide. From the in-storage processing
engine's point of view, the network looks like a variable number of FIFOs, all
of which can be declared to be of different bit-width according to the use. The
network FIFOs have features that are expected of normal FIFOs, such as back
pressure. The network FIFOs also act as virtual link endpoints. Such intuitive characteristics of the network should aid in the ease
of in-storage processor development. The use of the network looks like the following, in pseudocode:

\begin{verbatim}
NetworkEndpoint#(type Bit#(64)) endpoint1 
     <- mkNetworkEndpoint(1); // endpoint id: 1
NetworkEndpoint#(type Bit#(32)) endpoint2 
     <- mkNetworkEndpoint(8); // endpoint id: 8
List endpoints = 
     cons(endpoint1, cons(endpoint2, nil));
NetworkArbiter arbiter 
     <- mkNetworkArbiter(endpoints);

...
endpoint1.send(data,dest);

...
tuple(data,src) <- endpoint1.receive;
\end{verbatim}

In our FlashBoost implementation, this network is implemented using the
low-latency serial links that are already included in the FPGA. By implementing
routing in the hardware and using a very low-latency network fabric, we were
able to achieve very high performance, with 0.5$\mu s$ of latency per network
hop, and near 10Gbps of bandwidth per link. Our implementation has a network
fan-out of 8 ports per storage node, so the aggregate network bandwidth
available to a node is up to 10Gbps.

\subsubsection{Link Layer}

The link layer managed physical connections between network ports in the storage
nodes. The most important aspect of the link layer is the simple token-based
flow control implementation. This assures that no packet will drop if the data
rate is higher than what the network can manage, or if the data is not received
from the destination node quick enough.

\subsubsection{Routing Layer}

Figure~\ref{fig:networkinterface} shows the implementation of the router in each
storage node. Each packet can either come from the network port's link layer
interface, or from the user's network endpoint. Each incoming packet's
destination field is compared to the routing table in the router to determine
whether to be forwarded to a remote node via a network port, or to be delivered
to a local endpoint if its destination is the current node. It then goes through
one of the four crossbar switches to be delivered to the correct destination. 
Fairness is implemented using a round-robin priority ordering to ensure maximum
throughput while ensuring no port starves.

It is important to note that the routing table can have more than one network
port entry per destination node index. This is because each pair of nodes can
have more than one immediate link connecting between them, and there may be
multiple viable paths between a pair of two remote nodes. In order to make
maximum use of the available bandwidth in such cases, the routing table can have
entries for multiple valid ports for the next hop. In order to ensure in-order
arrival, the endpoint id of the origin endpoint is hashed to deterministically
decide which entry to use in the routing table. In-order arrival is desirable in
a hardware implementation because completion buffers may be expensive.
Because of this characteristic, if the in-storage processing engine want to
ensure more bandwidth, it can break a wide data bus into multiple endpoints and
send data over them in parallel.

It should be pointed out that because end-to-end flow control doesn't exist, a
large part of the network might actually block if a destination endpoint doesn't
receive the data quick enough. This was an intentional design choice to build an
extremely low-latency network while maintaining low resource usage. If the designer is sure that the packets will
always be consumed, end-to-end flow control can be omitted. Otherwise, the
designer can choose to implement a flow control protocol using the network
endpoints. We plan to provide pre-implemented flow control modules that can be
plugged in.

\begin{figure}[h]
	\begin{center}
	\includegraphics[scale=0.4]{figures/network-routing-crop.pdf}
	\caption{Network Interface}
	\label{fig:networkinterface}
	\end{center}
\end{figure}


\subsection{Host Interface}

The near-data processing core can be accessed from the host server over either a
low-level RPC-like interface or a file system abstraction. Our host interface
was implemented using Connectal~\ref{connectal}, a hardware-software codesign
framework built by Quanta Research Cambridge. Connectal reads the interface
definition file written by the programmer and generates glue logic between
hardware and software. Connectal provides an RPC-like interface, as well as a
memory-mapped DMA interface for high bandwidth data transfer.

The software maintains two 128-page page buffers, each for reads and writes. When
issuing a read or write request, the software sends the target or source page
buffer index along with the request to let the hardware side host interface know
where to read or write data from. Since this buffer is returned to the free list
when each operation is finished the buffer index effectively acts as the locally
unique tag for each flash operation. Writing data is straightforward, as data is
read in-order from the host page buffer and written in-order to each flash
controller bus. However, reading from flash is slightly more complex because the
data from each flash chip can come interleaved, even within the same bus. 
We include an implementation of a multiple-in-single-out completion buffer using
on-chip BRAM with burst support, in order to save hardware resources by not
having 128 separate FIFOs. Figure~\ref{fig:hostinterface} describes the
structure of the host interface.

\begin{figure}[ht!]
	\begin{center}
	\includegraphics[width=0.3\textwidth]{figures/readinterface-crop.pdf}
	\includegraphics[width=0.3\textwidth]{figures/writeinterface-crop.pdf}
	\caption{Host-FPGA Interface Over PCIe}
	\label{fig:hostinterface}
	\end{center}
\end{figure}

%TODO: \subsubsection{Storage Bridge to Host}

\subsection{File System Interface}

Sungjin!

\begin{figure}[h]
	\begin{center}
	\includegraphics[width=0.4\paperwidth]{figures/swstack.png}
	\caption{File System Interface}
	\label{fig:filesystem}
	\end{center}
\end{figure}


\begin{figure}[h]
	\begin{center}
	\includegraphics[width=0.15\textwidth]{resources/topology-crop.pdf}
	\caption{Prototype Topology}
	\label{fig:topology}
	\end{center}
\end{figure}

\section{Implementation Details}
\label{sec:implementation}

We have implemented a prototype of the network described using a cluster
machines. Each node in the cluster consists of one Intel Xeon-based server,
Xilinx VC707 FPGA development board, and a network expansion card. Each VC707
development board was augmented with a network expansion card that plugged into
the FMC expansion port, which pinned out the four GTX multi-gigabit serial
transceivers assigned to the high pin count FMC port into four SATA ports. SATA
crossover cables were used to connect the network expansion cards. Two lanes, or
two SATA cables were grouped together to form a channel. In the prototype system
the cluster was networked into a one-dimensional bar topology, depicted in
Figure~\ref{fig:topology}. Since the VC707 board has two FMC ports, each node
can have a fan out of up to 4 channels, which can be used to implement other
various kinds of topologies. Any direct network with a fan-out per node of 4 or
less can be implemented. Some examples of such topologies are depicted in
Figure~\ref{fig:topologies}.


\begin{figure}[h]
	\centering
	\subfloat[Star]
		{\includegraphics[scale=0.35]{resources/star-crop.pdf}}
		\hfill
	\subfloat[Tree]
		{\includegraphics[scale=0.35]{resources/tree-crop.pdf}}
		\hfill
	\subfloat[Mesh]
		{\includegraphics[scale=0.35]{resources/mesh-crop.pdf}}
	\caption{Any Network Topology With Less Than 4 Fan-Out is Possible}
	\label{fig:topologies}
\end{figure}


The link latency of an aurora link based on the GTX multi-gigabit transceiver
was measured to be around 0.48us, which translates to about 75 cycles on the
6.4ns user clock. The link layer flow control was implemented with a
conservative size of 200 beats for the round trip delay.

\subsection{FPGA Resource Utilization}

We have measured the FPGA resource utilization of our network using a simple
setup with two endpoints: one high speed endpoint with larger flow control
stride and buffer size (Stride of 200 and buffer size of 1024 packets), and one
small endpoint with smaller buffers. The endpoint row in the table below
described the larger endpoint. The router component includes the chaining logic
used to link the endpoints to it. The user logic was clocked at 125MHz.

\begin{table}[h]
	\begin{center}
	\begin{tabular}{l | c | c | }
		Component & LUTS & RAMB36 \\
		\hline
		\hline
		Aurora Link & 4843 & 36 \\
		Router & 3743 & 0 \\
		Endpoint ($\times$2) & 753 & 3 \\
		\hline
		\hline
		Total & 10092 & 42\\
		Virtex 7 Percentage   & (3\%) & (4\%) \\
	\end{tabular}
	\caption{FPGA Resource Utilization}
	\label{tab:fpgautil}
	\end{center}
\end{table}

\section{Results}
\subsection{Configurations}

\subsection{Distributed Flash Access Performance}
Latency for local flash reads, reads from local DRAM, remote reads over
fpga/serial, remote reads over serial/host DRAM, remote reads over
ethernet/Flash ethernet/DRAM

Measure latency of 8K page reads + filtering 

Aggregate bandwidth for flash, collecting all flash to one node, flash to pcie?
Aggregate bandwidth for competing workloads on multiple servers \(scalability\)



\section{Conclusion}
\label{sec:conclusion}

An efficient and high-performance network is essential for development of an
effective distributed FPGA computing platform. Due to high engineering cost and
the scarce on-chip memory resource, many existing inter-FPGA network
implementations do not provide transport-level network functionality such as
end-to-end flow control and virtual channels. Because such systems depend on the
user application developer to implement such features and write safe
applications, it becomes difficult to create complex distributed FPGA
applications which are also deadlock-safe and high performance.

In this paper, we have presented our design of a parameterized, low overhead
transport-layer network that provides useful features such as
virtual channels and end-to-end flow control. Our network takes advantage of the
high reliability of the high-speed serial links, which are integrated in the
FPGA fabric, to implement a lossless in-order network layer, which allowed us to
simplify the transport layer and use less FPGA resources. The design of the
transport layer is parameterized, so that the developer can choose to use less
resources while meeting the performance requirements of the individual endpoint.
Our prototype implementation demonstrated a high performance in an FPGA cluster
setting. We predict that our network will accelerate future research of
distributed FPGA applications.





% conference papers do not normally have an appendix


% use section* for acknowledgement
%\section*{Acknowledgment}
%The authors would like to thank...





% trigger a \newpage just before the given reference
% number - used to balance the columns on the last page
% adjust value as needed - may need to be readjusted if
% the document is modified later
%\IEEEtriggeratref{8}
% The "triggered" command can be changed if desired:
%\IEEEtriggercmd{\enlargethispage{-5in}}

% references section

% can use a bibliography generated by BibTeX as a .bbl file
% BibTeX documentation can be easily obtained at:
% http://www.ctan.org/tex-archive/biblio/bibtex/contrib/doc/
% The IEEEtran BibTeX style support page is at:
% http://www.michaelshell.org/tex/ieeetran/bibtex/
%\bibliographystyle{IEEEtran}
% argument is your BibTeX string definitions and bibliography database(s)
%\bibliography{IEEEabrv,../bib/paper}
%
% <OR> manually copy in the resultant .bbl file
% set second argument of \begin to the number of references
% (used to reserve space for the reference number labels box)

\bibliographystyle{IEEEtran}
\bibliography{IEEEabrv,references,network}

%\begin{thebibliography}{1}
%
%\bibitem{IEEEhowto:kopka}
%H.~Kopka and P.~W. Daly, \emph{A Guide to \LaTeX}, 3rd~ed.\hskip 1em plus
%  0.5em minus 0.4em\relax Harlow, England: Addison-Wesley, 1999.
%
%\end{thebibliography}




% that's all folks
\end{document}


