\section{Introduction}

As computation and communication technologies become increasingly advanced and
widespread, an unprecedented amount of digital data has become available to be
analyzed. In-depth analysis of so-called "Big Data" has shown to provide
extremely deep insight into real world phenomena.

Due to the sheer size of the datasets involved in Big Data analytics,
transporting data to where it can be processed becomes a first order concern. In
traditional computer systems, this meant first trying to fit as much data as
possible on local DRAM, and when this becomes not viable, creating a cluster of
machines connected by a fast network fabric. Storing data in secondary storage
was done as a last-ditch attempt to bring price down by sacrificing performance.
This is due to the disparity of performance across system components, where the
access latency of spinning disk storages dwarfed all other components of a
system/

%However, traditional computing machines are not designed to 

%More data means more value <- as much data as can process.

%Big Data by definition is data too large to be processed with conventional
%methods and machines.

This landscape is changing with the widespread adoption of flash storage. Flash
storage’s random access latency is multiple orders of magnitude lower than disk,
on par with most widespread networking platforms. As a result, a computing
system with fewer nodes storing data on a local flash storage can be an
alternative solution instead of a system with a greater number of nodes
attempting to store most data in DRAM. 

One way to further improve processing capabilities of flash storage based
systems is near-data computation. Because transporting the data to the processor
incurs a large communication overhead, it would be beneficial if computation can
be moved to a processing core on the storage itself, so that much lower latency
computation can take place. Operations such as filtering can be offloaded to
such a near-data processor so that less data has to be transported to the CPU,
conserving bandwidth and improving performance. 

Another way to improve performance could be to create a separate sideband
network between storage devices, in order to accelerate remote storage access.
Such a network can have a hardware-accelerated application-specific network
protocol, reducing network overhead. This may enable implementing a
high-performance distributed file system by reducing the latency of
communication between nodes. It may be possible to implement micro-servers on
storage devices themselves that service other servers in the cluster over this
high-speed network.

We propose to explore various architectural solutions to make maximum use of
flash device characteristics and improve data processing capabilities while
lowering cost. This includes flash-optimized storage management software stack,
near-data processing using hardware accelerators, and application-specific
sideband networks between storage devices accelerating application packets and
distributed file systems.
