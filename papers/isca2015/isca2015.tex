\documentclass[pageno]{jpaper}

%replace XXX with the submission number you are given from the ISCA submission site.
\newcommand{\iscasubmissionnumber}{XXX}

\usepackage{multirow}
\usepackage{listings}
\usepackage[normalem]{ulem}

\begin{document}

\title{
BlueDBM: An Appliance for Big Data Analytics
}

\author{
	Sang-Woo Jun$^\dagger$\quad Ming Liu$^\dagger$\quad Sungjin Lee$^\dagger$\quad Jamey
	Hicks$^\star$\quad \\
	John
	Ankcorn$^\star$\quad Myron King$^\star$\quad Shuotao Xu$^\dagger$\quad Arvind$^\dagger$ \\
	\\
	Department of Electrical Engineering and Computer Science \\
	Massachusetts Institute of Technology$^\dagger$ \\
	Quanta Research Cambridge$^\star$ \\
  \authemail{\{wjun,ml,chamdoo,shuotao,arvind\}@csail.mit.edu}$^\dagger$ \\
  \authemail{\{jamey.hicks,john.ankcorn,myron.king\}@qrclab.com}$^\star$
}
\date{}
\maketitle
%%%%%%%%%%%%%%%%%%%%%%%%%%%%%%%%%%%%%%%%%%%%%%%%%%%%%%%%%%%%%%%%%%%%% 
% The ACM Copyright Paragraph must appear on the first page of each 
% paper. Government authors should refer to the alternative copyright
% instructions @ http://www.acm.org/sigs/volunteer_resources/conference_manual/6-5proc
%%%%%%%%%%%%%%%%%%%%%%%%%%%%%%%%%%%%%%%%%%%%%%%%%%%%%%%%%%%%%%%%%%%%% 
 
\copyrightnotice{
Permission to make digital or hard copies of all or part of this work for
personal or classroom use is granted without fee provided that copies are not
made or distributed for profit or commercial advantage, and that copies bear
this notice and the full citation on the first page. To copy otherwise, to
republish, to post on servers or to redistribute to lists, requires prior
specific permission and/or a fee.\\
\textit{ISCA'15, June 13-17, 2015, Portland, OR USA}\\
Rights management text and bibliographic strip from ACM placed here.
}

\thispagestyle{empty}

\begin{abstract}
Complex data queries, because of their need for random accesses, have proven to
be slow unless all the data can be accommodated in DRAM. There are many domains,
such as genomics, geological data and daily twitter feeds where the datasets of
interest are 5TB to 20 TB. For such a dataset, one would need a cluster with 100
servers, each with 128GB to 256GBs of DRAM, to accommodate all the data in DRAM.
On the other hand, such datasets could be stored easily in the flash memory of a
rack-sized cluster. Flash storage has much better random access performance than
hard disks, which makes it desirable for analytics workloads. In this paper we
present BlueDBM, a new system architecture which has flash-based storage with
in-store processing capability and a low-latency high-throughput
inter-controller network.  We show that BlueDBM outperforms a flash-based system
without these features by a factor of 10 for some important applications. While
the performance of a ram-cloud system falls sharply even if only
5\%\textasciitilde10\% of the references are to the secondary storage, this
sharp performance degradation is not an issue in BlueDBM. BlueDBM presents an attractive point in the cost-performance trade-off for Big Data analytics. 
\end{abstract}


\section{Introduction}
\label{sec:intro}

Google has predicted flu outbreaks by analyzing social network information a
week faster than CDC~\cite{googleflu}. Analysis of twitter data can reveal social
upheavals faster than journalists. Amazon is planning to use customer data for
preemptive shipping of products. Real-time analysis of personal genome may
significantly aid in diagnostics. Big Data analytics are potentially going to
have revolutionary impact on the way scientific discoveries are made. By many
accounts, complex analysis of Big Data is going to be the biggest economic
driver for the IT industry.

Big Data by definition doesn’t fit in personal computers or DRAM of even
moderate size clusters. Since the data may be stored on hard disks, latency and
throughput of storage access is of primary concern. Historically, this has been
mitigated by organizing the processing of data in a highly sequential manner.
However, complex queries cannot always be organized for sequential data
accesses, and thus high performance implementations of such queries pose a
great challenge. One approach to solving this problem is \emph{ram
cloud}~\cite{ramcloud}, where the cluster has enough collective DRAM to accommodate the
entire dataset in DRAM. In this paper, we explore much cheaper alternatives
where Big Data analytics can be done with reasonable efficiency in a single
rack.

Another alternative to speed up complex data analysis is to use SSDs instead of
disks because of superior performance of flash devices in terms of latency of
access and throughput, especially for random accesses. SSDs have been developed
with the goal to be a drop-in replacement of hard disks. This has led to their
widespread use especially in embedded space, but the use of this legacy
interface has resulted in suboptimal use of flash capabilities. The
sub-optimality arises both from the use of a \emph{Flash Translation Layer}
(FTL), and the use of old hardware interfaces like SATA. The latter is somewhat
mitigated by the use of newer interfaces like PCIe.  High-performance enterprise
SSDs such as FusionIO~\cite{fusionio}, Violinmemory~\cite{violinmemory} and Intel NVMe
devices~\cite{intelnvme} solve many of these issues by implementing an improved interface
such as NVMe over PCIe. Attempts to remove the FTL and let the database make
high level decisions~\cite{noftl} have shown to be beneficial, but such
solutions are not yet widely adopted.


The latency to access storage over Ethernet in a hard-disk-based cluster is
dominated by the latency of the hard disk itself. However, this is not the case
in an SSD-based cluster, where network latency may even be larger than the SSD
latency. At this low-level of latency, even software stack overhead becomes a
significant concern. These concerns have been addressed by faster network
fabrics such as 10Gb Ethernet and Infiniband~\cite{infiniband}, and by low-overhead
software protocols such as RDMA~\cite{rdmampi} or user-level TCP stacks that bypass the
operating system~\cite{usertcp}. QuickSAN integrates a network interface into the
storage device, thus removing a layer of software overhead.

Another solution to the network performance problem is to reduce the network
traffic by in-store computing. For example, if one can perform some filtering
operation in the disk controller without bringing it to the host, it may
dramatically reduce the amount of data that needs to be transferred between the
disk and the host~\cite{idisk,netezza,smartssdquery}. Even though this idea has been
around for a long time, it has not found much traction, perhaps because the
characteristics of disks masked the benefits of this approach.  

In this paper, we present FlashBoost, a system designed to address all of the
aforementioned problems in the context of complex analysis of Big Data. Our goal
is to provide a rack-level system which can address many Big Data problems whose
datasets are 10\textasciitilde20 TB. Specifically, we have implemented a system which has 20
nodes where each node consists of a server with 12 Xeon cores, 48 GB of DRAM and
2 TB of disk. We have augmented each server with a Xilinx VC707 FPGA board and
1TB of flash on a custom design board. The FPGA board is connected to the server
via PCIe on one side, and to the flash card by 8 lanes of 6.6Gbps serial
communication links. All FPGAs can be connected to each other in various
topologies using 8 10Gbps serial links provided by the FPGA. The overall
architecture can be seen in Figure~\ref{fig:architecture}.

The system hardware and software provides the following capabilities:
\begin{enumerate}
\item Large enough storage to host Big Data workloads in the 10\textasciitilde20 TB range
\item Near-uniform latency access into a network of storage devices that form a
global address space
\item Capacity to implement user-defined in-storage processing engines in the
embedded FPGA
\item Flash card design which exposes a low-latency interface with details to
exploit parallelism in flash chip accesses by in-storage processing engines
\end{enumerate}

Our preliminary experimental results show that FlashBoost performance is
significantly better than (1) a disk-based system and (2) a flash-based system
in which flash is used as a disk replacement. We also show that accelerators can
provide a factor of XXX performance benefits over a non-accelerated system.
FlashBoost unambiguously establishes an architecture whose
price-performance-power characteristics provide an attractive alternative for
doing similar scale applications in a ram cloud.

The main contributions of this work are: (1) Design and implementation of a
scalable flash-based system with a global address space and in-store computing
capability. (2) A hardware-software codesign environment for incorporating
user-defined in-store processing engines. (3) Performance measurements that show
the advantage of such an architecture over using flash as a drop-in replacement
for disks. (4) Demonstration of a complex data analytics appliance which is much
cheaper and consumes an order of magnitude less power than the cloud-based
alternative.

The rest of the paper is organized as follows: In section~\ref{sec:related} we
explore some existing research related to our system. In
section~\ref{sec:architecture} we describe the architecture of our rack-level
system, and in section~\ref{sec:software} we describe the software interface
that can be used to access flash and the accelerators. In
section~\ref{sec:acceleration} we describe some example accelerators we have
built for the FlashBoost system. In section~\ref{sec:implementation} we describe
our hardware implementation of FlashBoost, and show our results from the
implementation in section~\ref{sec:results} and \ref{sec:results_acceleration}.

%\section{Motivation}

Advancement of high-performance flash storage is introducing a new dynamic into
how computer systems can be designed.  In traditional computer systems, the
amount of data that can be processed on a single machine is often limited by the
fast random access memory, i.e., DRAM, available on it. This is because the
random access latency of mechanical hard disk, which is the primary form of
secondary storage is large enough to dwarf all other components in the system.
Because if this, the only way to quickly process large amounts of data was to
store it in the collective memory of a cluster of machines.

This landscape is changing with the widespread adoption of flash storage. Flash
storage's random access latency is multiple orders of magnitude lower than disk,
on par with many widespread networking interfaces. This means in a distributed
cluster, it may take as much time to access and process a piece of data in the
dram of a remote machine as it may take to fetch and process data on a local
flash storage, especially if it is possible to bypass the software overhead
imposed on flash storage by the operating system to make it compatible with the
generic disk access interface.

One of the reasons a generic network link has such a high latency is because of
the general-purpose protocol overhead. Because general-purpose network protocols
such as TCP need to cater to all kinds of traffic, the protocol
becomes very complex and the software stack creates a large overhead. There have
been many different attempts in reducing this overhead, ranging from bypassing
the network stack and implementing the protocol in the userspace, to
implementing parts of the protocol in the network interface hardware itself.

One way to reduce latency and also make more efficient use of bandwidth is to
embed computation engines directly in the storage and network controller itself.
Latency can be cut down by eliminating a data transport step from the
storage or network controller to the CPU, and bandwidth amplification can be
achieved by implementing compression or filtering operations in the controllers.

Application-specific hardware accelerators might be a good fit for the embedded
computation engines. Hardware accelerators implemented as a standalone
appliances are usually optimized for throughput rather than latency, as the
invocation latency over PCIe or other interface medium is already substantial.
On the other hand, low-latency hardware accelerators in the form of augmented
network interfaces have been successful in many low-latency workloads such as
high-frequency trading.



\section{Related Work}

%\item Computation/Storage coupling (computation near data), for MapReduce, etc
%\item Flash-based databases (Zetascale)
Flash-based databased have been under active investigation as an way to expand
the working set beyond what is feasible using relatively expensive DRAM.
Flash-based solutions such as SanDisk ZetaScale~\ref{zetascale} have
demonstrated that by optimizing databases for flash access, a
performance-per-dollar much higher that with DRAM-based machines can be
achieved.

As flash storage became faster, and the bottleneck of system performance became
other components such as the storage area network, attempts to reduce the
network software overhead or even providing a separate sideband network is being
explored. One such solution is QuickSAN~\ref{quicksan}, which implements much of
the network protocol in a reconfigurable hardware fabric.

FPGA-based application-specific hardware accelerators is becoming more popular
for HPC workloads due to its high performance and low power consumption.
Applications such as linear algebra operations~\ref{}, genome sequencing~\ref{}
and high-speed data filtering~\ref{} has been successful, and even general
purpose database accelerators have been built and commercialized~\ref{netezza}.
Microsoft Research recently has demonstrated the power/performance
benefits of using FPGAs in a cluster, in the context of accelerating Bing
search~\ref{msr}. In their work, each server in a cluster were augmented with an FPGA, and
the FPGAs were networked to each other using high-speed serial links.
However, such accelerator appliances depend on the host streaming data to the
accelerator, and therefore are only effective for datasets that can
still fit in the combined DRAM capacity of host machines. Once the data doesn't
fit, data needs to be streamed from secondary storage, which ends up being the
bottleneck.

Another way hardware accelerators can be deployed is an embedded processor in
the datapath of normal data access, such as in network and storage interfaces. 
An accelerator with an implementation of the
network protocol can plug directly into a network
port, and process data as it is being passed across the network link. This
removes much of the data transport and protocol overhead and allows very low
latency operations. This idea has often been demonstrated in financial
applications that require extremely low latency, and also in latency-sensitive
applications such as memcached~\ref{memcached}. Accelerators such as compression
or filtering accelerators in the network or storage can help reduce the amount
of data that need to be sent over a link, effectively amplifying its bandwidth.

\section{System Architecture}

The FlashBoost architecture is composed of multiple identical nodes. Each node
consists of a host server and a FlashBoost storage device. The host servers are
networked together using Ethernet or other general-purpose networking fabric.
The FlashBoost storage device consists of a in-store processing engine, flash
storage and flash controller, host interface, and a high-speed network controller.
The in-store processing engine can implement a raw
access to flash and network resources, an application-specific hardware
accelerator, or both. The host software can send high-level requests over the
PCIe link to the in-storage processing engine to access flash network and computation resources.
In our implementation of FlashBoost, we have used a Field Programmable Gate
Array (FPGA) to implement The in-store processor and also the flash, host and
network controllers. However, the FlashBoost Architecture should not be limited
to an FPGA-based implementation.


\begin{figure*}[ht]
	\begin{center}
	\includegraphics[width=0.8\paperwidth]{figures/architecture.pdf}
	\caption{FlashBoost Architecture}
	\label{fig:architecture}
	\end{center}
\end{figure*}


The in-store processing engine has access to four major services, The flash
controller, network controller, host interface and the DRAM Buffer.
Figure~\ref{fig:ispservice} shows the four services. Development of FlashBoost was
done in the high-level hardware description language Bluespec. As a result, all
services expose a high-level language interface using latency-insensitive FIFOs
for communication. This makes the services intuitive to use, and flexible to be
used easily with many in-storage processing engines.

\begin{figure}[h]
	\begin{center}
	\includegraphics[width=0.3\paperwidth]{figures/isp-service-crop.pdf}
	\caption{Services Provided to In-Store Processor}
	\label{fig:ispcore}
	\end{center}
\end{figure}

\subsection{Flash Interface}

The near-data processing core implements a error-free low-level interface into
the flash storage. (Ming)

\begin{figure}[h]
	\begin{center}
	\includegraphics[scale=0.4]{figures/top-arch-crop.pdf}
	\caption{Flash Interface}
	\label{fig:flashinterface}
	\end{center}
\end{figure}

\subsection{Inter-Storage Network}

Conventional general-purpose networking infrastructures such as TCP are designed to serve
all kinds of network traffic over any distance, ranging from intra-rack to over
continents. This makes the protocol incredibly complex and results in a large
protocol overhead.

The near-data processors in FlashBoost are linked together using a separate
high-performance network among themselves in which all nodes have multiple
network ports and act both as a network switch as well as an endpoint. This network provides low-level
routing, some flow control and use of virtual channels, but omits various high level
features that modern network protocols provide. From the in-storage processing
engine's point of view, the network looks like a variable number of FIFOs, all
of which can be declared to be of different bit-width according to the use. The
network FIFOs have features that are expected of normal FIFOs, such as back
pressure. The network FIFOs also act as virtual link endpoints. Such intuitive characteristics of the network should aid in the ease
of in-storage processor development. The use of the network looks like the following, in pseudocode:

\begin{verbatim}
NetworkEndpoint#(type Bit#(64)) endpoint1 
     <- mkNetworkEndpoint(1); // endpoint id: 1
NetworkEndpoint#(type Bit#(32)) endpoint2 
     <- mkNetworkEndpoint(8); // endpoint id: 8
List endpoints = 
     cons(endpoint1, cons(endpoint2, nil));
NetworkArbiter arbiter 
     <- mkNetworkArbiter(endpoints);

...
endpoint1.send(data,dest);

...
tuple(data,src) <- endpoint1.receive;
\end{verbatim}

In our FlashBoost implementation, this network is implemented using the
low-latency serial links that are already included in the FPGA. By implementing
routing in the hardware and using a very low-latency network fabric, we were
able to achieve very high performance, with 0.5$\mu s$ of latency per network
hop, and near 10Gbps of bandwidth per link. Our implementation has a network
fan-out of 8 ports per storage node, so the aggregate network bandwidth
available to a node is up to 10Gbps.

\subsubsection{Link Layer}

The link layer managed physical connections between network ports in the storage
nodes. The most important aspect of the link layer is the simple token-based
flow control implementation. This assures that no packet will drop if the data
rate is higher than what the network can manage, or if the data is not received
from the destination node quick enough.

\subsubsection{Routing Layer}

Figure~\ref{fig:networkinterface} shows the implementation of the router in each
storage node. Each packet can either come from the network port's link layer
interface, or from the user's network endpoint. Each incoming packet's
destination field is compared to the routing table in the router to determine
whether to be forwarded to a remote node via a network port, or to be delivered
to a local endpoint if its destination is the current node. It then goes through
one of the four crossbar switches to be delivered to the correct destination. 
Fairness is implemented using a round-robin priority ordering to ensure maximum
throughput while ensuring no port starves.

It is important to note that the routing table can have more than one network
port entry per destination node index. This is because each pair of nodes can
have more than one immediate link connecting between them, and there may be
multiple viable paths between a pair of two remote nodes. In order to make
maximum use of the available bandwidth in such cases, the routing table can have
entries for multiple valid ports for the next hop. In order to ensure in-order
arrival, the endpoint id of the origin endpoint is hashed to deterministically
decide which entry to use in the routing table. In-order arrival is desirable in
a hardware implementation because completion buffers may be expensive.
Because of this characteristic, if the in-storage processing engine want to
ensure more bandwidth, it can break a wide data bus into multiple endpoints and
send data over them in parallel.

It should be pointed out that because end-to-end flow control doesn't exist, a
large part of the network might actually block if a destination endpoint doesn't
receive the data quick enough. This was an intentional design choice to build an
extremely low-latency network while maintaining low resource usage. If the designer is sure that the packets will
always be consumed, end-to-end flow control can be omitted. Otherwise, the
designer can choose to implement a flow control protocol using the network
endpoints. We plan to provide pre-implemented flow control modules that can be
plugged in.

\begin{figure}[h]
	\begin{center}
	\includegraphics[scale=0.4]{figures/network-routing-crop.pdf}
	\caption{Network Interface}
	\label{fig:networkinterface}
	\end{center}
\end{figure}


\subsection{Host Interface}

The near-data processing core can be accessed from the host server over either a
low-level RPC-like interface or a file system abstraction. Our host interface
was implemented using Connectal~\ref{connectal}, a hardware-software codesign
framework built by Quanta Research Cambridge. Connectal reads the interface
definition file written by the programmer and generates glue logic between
hardware and software. Connectal provides an RPC-like interface, as well as a
memory-mapped DMA interface for high bandwidth data transfer.

The software maintains two 128-page page buffers, each for reads and writes. When
issuing a read or write request, the software sends the target or source page
buffer index along with the request to let the hardware side host interface know
where to read or write data from. Since this buffer is returned to the free list
when each operation is finished the buffer index effectively acts as the locally
unique tag for each flash operation. Writing data is straightforward, as data is
read in-order from the host page buffer and written in-order to each flash
controller bus. However, reading from flash is slightly more complex because the
data from each flash chip can come interleaved, even within the same bus. 
We include an implementation of a multiple-in-single-out completion buffer using
on-chip BRAM with burst support, in order to save hardware resources by not
having 128 separate FIFOs. Figure~\ref{fig:hostinterface} describes the
structure of the host interface.

\begin{figure}[ht!]
	\begin{center}
	\includegraphics[width=0.3\textwidth]{figures/readinterface-crop.pdf}
	\includegraphics[width=0.3\textwidth]{figures/writeinterface-crop.pdf}
	\caption{Host-FPGA Interface Over PCIe}
	\label{fig:hostinterface}
	\end{center}
\end{figure}

%TODO: \subsubsection{Storage Bridge to Host}

\subsection{File System Interface}

Sungjin!

\begin{figure}[h]
	\begin{center}
	\includegraphics[width=0.4\paperwidth]{figures/swstack.png}
	\caption{File System Interface}
	\label{fig:filesystem}
	\end{center}
\end{figure}

\section{Software Interface}
\label{sec:software}

The primary design goals of the FlashBoost software are to manage NAND flash,
hiding its physical properties,
while offering easy-to-use interfaces 
between user-level applications and hardware accelerators in the storage device.

NAND flash has very different characteristics from traditional hard disk drives,
so a flash-aware management layer like a Flash Translation Layer (FTL)~\cite{}
is required to hide such differences from the rest of the system.
To this end, we employed a refactored I/O architecture
that offloads almost all of the FTL functions into a log-structured file system, called RFS~\cite{}.
Unlike conventional FTL designs where the flash characteristics are
hidden from the file system, RFS performs some functionality of an FTL,
including logical-to-physical address mapping and garbage collection.
This helps achieve achieving better garbage collection
efficiency at much lower memory requirement. For the sake of compatibility with
existing software, FlashBoost also offers a full-fledged page-level FTL for
end-users, which is implemented in the device driver, much like Fusion IO’s
driver implementation. It allows us to use well-known Linux file systems (e.g., ext2/4/3) 
as well as database systems (directly running on top of a block device) with FlashBoost.

%The most important feature of the FlashBoost software stack is that it provides
%easy-to-use software interfaces for application-specific hardware accelerators, 
%allowing developers to easily make use of fast near-data processing 
%without any efforts to write their own custom interfaces manually. 
The FlashBoost software allows developers to easily make use of fast near-data processing 
without any efforts to write their own custom interfaces manually. 
Figure~\ref{fig:filesystem} shows how user-level applications access hardware accelerators.
In the FlashBoost software stack, user-level
applications can provide the near-data processing engines with the physical
locations of data in the flash. 
Such physical locations can be easily obtained from the file
system (i.e., RFS) or the device driver, because they maintain the mapping
information between user files and flash storage (see (1) in Figure~\ref{fig:filesystem}). 
User-level applications then send queries together with physical addresses (2).
The in-store processor reads data from NAND flash directly and performs its analytic jobs (3).
Once the in-storage processor
finishes processing on the provided data, it will write the results to a host
memory region and notify the user application (4).

Since data analytics are done by well-optimized application-specific hardware and 
actual data transfers between the host and the storage device are avoided,
it greatly improves the performance of data analytics.
It is worth noting that, in FlashBoost,
all the user requests, including both user queries and data, are sent to the hardware directly, 
bypassing almost all OS kernel, except for essential driver modules. 
This helps us to avoid deep OS kernel stacks that often cause long I/O latencies. 
 
It is also very common that multiple
user-applications compete for the limited and precious hardware acceleration
units. For efficient sharing of hardware resources, FlashBoost runs a scheduling
daemon that assigns available hardware-acceleration units to competing
user-applications. In our implementation, a simple FIFO-based policy is used for
request scheduling.

\begin{figure}[h]
	\begin{center}
	\includegraphics[width=0.4\paperwidth]{figures/software.pdf}
	\caption{File System Interface}
	\label{fig:filesystem}
	\end{center}
\end{figure}

\section{Hardware Implementation}

We have built a 20-node BlueDBM cluster to explore the capabilities of the
architecture. Figure~\ref{fig:bluedbmcluster} shows the photo of our
implementation.

\begin{figure}[ht]
	\begin{center}
	\includegraphics[width=0.3\paperwidth]{figures/rack.jpg}
	\caption{A 20-node BlueDBM Cluster}
	\label{fig:bluedbmcluster}
	\end{center}
\end{figure}

The cluster consists of 20 Xeon servers mounted in two racks, each with a Xilinx
VC707 FPGA development board connected via a PCIe connection. Each VC707 board
hosts two custom-built flash boards with SATA connectors. The VC707 board,
coupled with two custom flash boards is mounted on top of each server.
The host servers run the Ubuntu distribution of Linux.
Figure~\ref{fig:bluedbmnode} shows the components of a single node.

\begin{figure}[ht]
	\begin{center}
	%\includegraphics[width=0.4\paperwidth]{figures/flashboard.jpg}
	\caption{A BlueDBM Node}
	\label{fig:bluedbmnode}
	\end{center}
\end{figure}

\subsection{Custom Flash Board}

\begin{figure}[ht]
	\begin{center}
	\includegraphics[width=0.4\paperwidth]{figures/flashboard.jpg}
	\caption{A Custom-Built Flash Board with 512GB of NAND Flash}
	\label{fig:flashboard}
	\end{center}
\end{figure}

We have designed and built a high-capacity custom flash board with high-speed
serial connectors, with the help of Quanta Inc., and Xilinx Inc.

Each flash card has 512GBs of NAND flash storage and a Xilinx Artix 7 chip, and
plugs into the host FPGA development board via the FPGA Mezzanine Card (FMC)
connector. The flash controller and Error Correcting Code (ECC) is implemented
on this Artix chip, providing the Virtex 7 FPGA chip on the VC707 a logical
error-free access into flash. The flash board also hosts 8 SATA connectors, 4 of
which pin out the high-speed serial ports on the host Virtex 7 FPGA,
and 4 of whch pin out the high-speed serial ports on the Artix 7 chip.
The serial ports are capable of 10Gbps and 6.6Gbps of bandwidth, respectively.

\section{Results}
\subsection{Configurations}

\subsection{Distributed Flash Access Performance}
Latency for local flash reads, reads from local DRAM, remote reads over
fpga/serial, remote reads over serial/host DRAM, remote reads over
ethernet/Flash ethernet/DRAM

Measure latency of 8K page reads + filtering 

Aggregate bandwidth for flash, collecting all flash to one node, flash to pcie?
Aggregate bandwidth for competing workloads on multiple servers \(scalability\)

\section{Application Acceleration}
\subsection{Hardware-Accelerated Grep}

\subsection{Sparse Matrix Operations}

%\section{Acceleration Performance Evaluation}
\label{sec:results_acceleration}

\begin{figure*}[ht]
\centering
\vspace{0pt}
\begin{minipage}[c]{.3\textwidth}
	\includegraphics[width=0.25\paperwidth]{graphs/obj/hammingfull-crop.pdf}
	\caption{Nearest Neighbor with FlashBoost up to Two Nodes}
	\label{fig:result_hammingfull}
\end{minipage}\hfill
\vspace{0pt}
\begin{minipage}[c]{.3\textwidth}
	\includegraphics[width=0.25\paperwidth]{graphs/obj/hamming-crop.pdf}
	\caption{Nearest Neighbor with Single-Node Throttled FlashBoost}
	\label{fig:result_hamming}
\end{minipage}\hfill
\vspace{0pt}
\begin{minipage}[c]{.3\textwidth}
	\includegraphics[width=0.25\paperwidth]{graphs/obj/graph-crop.pdf}
	\caption{Graph Traversal Performance}
	\label{fig:result_graph}
\end{minipage}
\end{figure*}

\subsection{Nearest Neighbor Search}

Figure~\ref{fig:result_hamming} shows the performance of nearest-neighbor search
with various data sources, normalized to the in-storage processing performance.
We compared FlashBoost against a high-cost fully DRAM configuration, as well as
realistic systems where some data cannot fit in DRAM.
Table~\ref{tab:nearest_neighbor} describes the benchmarks depicted in
Figure~\ref{fig:result_hamming}.

It should be noted that we have throttled our flash storage throughput to
600MB/s for this experiment, which is the bandwidth of the SATA 3.0
specification. This is to compare only the benefits of the in-store processing
architecture against other designs, otherwise the high bandwidth of the
FlashBoost hardware will result in an unfair comparison. When using all of the
2.4GB/s of our flash bandwidth, FlashBoost with ISP outperforms DRAM up to 4 threads.

\begin{tabular}{l | p{0.25\paperwidth}}
\label{tab:nearest_neighbor}
Name & Description \\
\hline \hline
DRAM & Store all data in DRAM \\
ISP & Process data in in-storage accelerator \\
FlashBoost+SW & Use FlashBoost as raw storage \\
Seq Flash & All requests are sequential flash accesses \\
10\% Flash & Store most data in DRAM. 10\% chance of hitting flash \\
5\% Disk & Store most data in DRAM. 5\% chance of hitting disk \\
Full Flash & All requests go to flash \\
\hline
\end{tabular}

It can be seen that streaming data directly from DRAM is obviously the fastest, and
scales linearly with thread count because it becomes a computation-bound
workload. The configuration that uses an in-storage processor to offload
computation is consistently faster than the software implementation, because
there is no software overhead involved, and the in-storage processor can process
the data at wire speed. Since sequential flash access outperforms FlashBoost
with two threads, it can be seen that this performance difference is not because
of the flash device performance but because of architectural differences and
software overhead.

Using the full bandwidth of the storage system would
have made this gap even more pronounced, as the software's bandwidth would be
limited by the PCIe running at 1.6GB/s. It can be seen that when even most of
the data can fit in DRAM, even rare access into storage can have a significant
impact on performance. These results further reinforces our claim that better
storage systems are required for effective analytics of very large datasets.

Figure~\ref{fig:result_hammingfull} shows the performance of un-throttled
FlashBoost, compared to a DRAM implementation. A single node of FlashBoost
outperforms DRAM up to 4 threads, and two nodes outperform DRAM up to 16
threads.

\subsection{Graph Traversal}

Figure~\ref{fig:result_graph} shows the performance of the graph traversal
accelerator, compared to a software implementation that accesses remote nodes
over a separate network. It can be seen that the low latency access to flash
becomes beneficial.




\subsection{Hardware-Accelerate String Search}

We were able to saturate the bandwidth of the flash store using multiple string
search cores.

This section is being written by Ming and Sungjin



\section{Conclusion and Future Work}
\label{sec:conclusion}

We have presented BlueDBM, an appliance for Big Data analytics that uses
flash storage, in-store processing and integrated networks for cost-effective
analytics of large datasets. A rack-size BlueDBM system is likely to be an
order of magnitude cheaper and less power hungry than a cloud based system with
enough DRAM to accommodate 10TB to 20TB of data. We have demonstrated the
performance benefits of BlueDBM using simple examples on large amounts of
data in comparison to a generic flash-based system without such architectural
improvements. We have also shown that the performance of a system which relies
on data being resident in DRAM, falls rapidly if even a small fraction of data
has to reside in secondary storage. BlueDBM like architecture does not suffer
from this problem because flash based systems with 10TB to 20TB of storage are
very affordable.



Our current implementation uses an FPGA to implement most of the new
architectural features, that is, in-store processors, integrated network
routers, flash controllers. It is straightforward to implement most of these
features using ASICs and provide some in-store computing capability via
general-purpose processors. This will simultaneously improve the performance
and lower the power consumption even further. Notwithstanding such developments
we are developing tools to make it easy to develop in-store processors for the
reconfigurable logic inside BlueDBM.  



We are currently developing or planning to develop several new applications
including:  
\emph{SQL Database Acceleration} by offloading query processing and
filtering to in-store processors,
\emph{Sparse-Matrix Based Linear Algebra Acceleration} and
\emph{BlueDBM-Optimized MapReduce}, which attempts to optimize data
flow of MapReduce to best fit an SSD-based cluster with in-store processors.
We plan to collaborate with other research groups to explore more applications.

%FIXME the intel code is for the big data project. Is this the correct one?
\section{Acknowledgements}
This work was partially funded by Quanta (Agmt. Dtd. 04/01/05), Samsung (Res.
Agmt. Eff. 01/01/12), Lincoln Laboratory (PO7000261350), and Intel (Agmt. Eff.
07/23/12). We also thank Xilinx for
their generous donation of VC707 FPGA boards and FPGA design expertise.

\vfill
%\pagebreak

\bstctlcite{bstctl:etal, bstctl:nodash, bstctl:simpurl}
\bibliographystyle{IEEEtranS}
\bibliography{references}

\end{document}


