
\section{Related Work}

%\item Computation/Storage coupling (computation near data), for MapReduce, etc
%\item Flash-based databases (Zetascale)
Flash-based databased have been under active investigation as an way to expand
the working set beyond what is feasible using relatively expensive DRAM.
Flash-based solutions such as SanDisk ZetaScale~\ref{zetascale} have
demonstrated that by optimizing databases for flash access, a
performance-per-dollar much higher that with DRAM-based machines can be
achieved.

As flash storage became faster, and the bottleneck of system performance became
other components such as the storage area network, attempts to reduce the
network software overhead or even providing a separate sideband network is being
explored. One such solution is QuickSAN~\ref{quicksan}, which implements much of
the network protocol in a reconfigurable hardware fabric.

FPGA-based application-specific hardware accelerators is becoming more popular
for HPC workloads due to its high performance and low power consumption.
Applications such as linear algebra operations~\ref{}, genome sequencing~\ref{}
and high-speed data filtering~\ref{} has been successful, and even general
purpose database accelerators have been built and commercialized~\ref{netezza}.

Microsoft Research recently has built and demonstrated the power/performance
benefits of a FPGA-based system called Catapult.
using FPGAs in a cluster, in the context of accelerating Bing
search~\ref{msr}. In their work, each server in a cluster were augmented with an FPGA, and
the FPGAs were networked to each other using high-speed serial links.

However, such accelerator appliances depend on the host streaming data to the
accelerator, and therefore are only effective for datasets that can
still fit in the combined DRAM capacity of host machines. Once the data doesn't
fit, data needs to be streamed from secondary storage, which ends up being the
bottleneck.

Another way hardware accelerators can be deployed is an embedded processor in
the datapath of normal data access, such as in network and storage interfaces. 
An accelerator with an implementation of the
network protocol can plug directly into a network
port, and process data as it is being passed across the network link. This
removes much of the data transport and protocol overhead and allows very low
latency operations. This idea has often been demonstrated in financial
applications that require extremely low latency, and also in latency-sensitive
applications such as memcached~\ref{memcached}. Accelerators such as compression
or filtering accelerators in the network or storage can help reduce the amount
of data that need to be sent over a link, effectively amplifying its bandwidth.


Ming's rant: database accelerators, old near data storage papers

The idea of near-disk or near-storage processing for databases is not new. In
1998, Keeton et al.~\cite{?} theorized about the idea of an intelligent disk
(IDISK) that integrated a cheap, low power embedded processor with DRAM and
hard disk.  These IDISKs are also connected to each other using serial network
switches.  Ideas such as using a shared intelligent disk controller, a
processor per head or per track to do simple scans and filters have been
suggested in the 1970s~\cite{?,?,?}.  However, at the time, the cost of these
special purpose hardware did not justify their performance gains. In
FlashBoost, we make use of programmable hardware, that have become widely
available, to provide flexibility in accelerating a variety of applications at
low cost, while easily keeping up with the data line rate of the flash buses. 

Building specialized hardware for databases has been studied extensively and
productized. Netezza~\cite{?} and Teradata~\cite{?} have used FPGAs as an
off-datapath accelerator to offload simple query operations. More recently,
domain specific processors for database queries were developed, including
Q100~\cite{?} and LINQits~\cite{?}. Q100 is a data-flow style processor with an
instruction set architecture that supported SQL queries. LINQits mapped a query
language called LINQ to a set of accelerated hardware templates on a
heterogeneous SoC (FPGA + ARM). Both designs exhibited order of magnitude
performance gains at lower power, affirming that specialized hardware for data
processing is very advantageous. However, in both cases, experiments were
conducted with data that is on-chip or in DRAM. In FlashBoost, we extend the
study by assuming that data does not fit in DRAM and we introduce an
architecture for storing and processing data in distributed flash. 

TODO: This may be too much... 

Accelerating SQL queries in-datapath was studied by Woods et al.~\cite{?}.
Ibex is a MySQL accelerator platform with a SATA SSD connected to an FPGA and
then connected to ethernet. Relational operators such as selection and group-by
are performed on the FPGA whenever possible, otherwise they are forwarded to
software. The FlashBoost architecture not only can accomodate existing
accelerators such as Ibex, it exposes internal flash device architecture and
provides a dedicated serial network for additional acceleration opportunities. 



